%%%%%%%%%%%%%%%%%%%%%%%%%%%%%%%%%%%%%%%%%
% ADAPTED VERSION
% 
% http://www.LaTeXTemplates.com
% Original author:
% Lorenzo Pantieri (http://www.lorenzopantieri.net) with extensive modifications by:
% Vel (vel@latextemplates.com)
%
% License:
% CC BY-NC-SA 3.0 (http://creativecommons.org/licenses/by-nc-sa/3.0/)
% 
% EDITED BY 
% Stella Grasshof
%%%%%%%%%%%%%%%%%%%%%%%%%%%%%%%%%%%%%%%%%
%----------------------------------------------------------------------------------------
%	REQUIRED PACKAGES
%----------------------------------------------------------------------------------------
\usepackage[T1]{fontenc} % Use 8-bit encoding that has 256 glyphs
\usepackage[utf8]{inputenc} % Required for including letters with accents
\usepackage{graphicx} % Required for including images
%\usepackage{placeins}
\usepackage{amsmath,amssymb,mathtools,bm,bbm} % For including math equations, theorems, symbols, etc
\usepackage{eurosym}
%\usepackage{pifont,gensymb} % symbol: \ding{<NUM>}
\usepackage{booktabs} % for nice tables with \toprule, \midrule
\usepackage{csvsimple,longtable}
\usepackage{enumitem}  % Required for manipulating the whitespace between and within lists
\usepackage{lipsum}    % Used for inserting dummy 'Lorem ipsum' text into the template
\usepackage{subfig}    % Required for creating figures with multiple parts (subfigures)
%\usepackage[format=hang]{subfig}
%\usepackage{varioref} % More descriptive referencing
\usepackage{color,xcolor}
\usepackage{algorithm,algpseudocode,float}
\usepackage{listings} % for source code stuff
\usepackage{setspace}
%\usepackage{lscape}
%\usepackage[authoryear]{natbib}
%\usepackage{comment} % comment out block wise

%----------------------------------------------------------------------------------------
%	TIKZ (if needed)
%---------------------------------------------------------------------------------------

%\usepackage{tikz,pgfplots}
%\usetikzlibrary{calc,shapes,arrows, intersections, tikzmark,fit,automata,positioning,quotes,snakes} 

%----------------------------------------------------------------------------------------
%	LANGUAGE SETTINGS
%---------------------------------------------------------------------------------------
%\usepackage[dutch]{babel}    % comment out if you write your thesis in Dutch
%\usepackage[english]{babel}  
%\usepackage[english,russian,main=french]{babel}
\usepackage[ngerman,main=english]{babel} % adapt to your needs

%----------------------------------------------------------------------------------------
%	HYPERLINKS SETTINGS
%---------------------------------------------------------------------------------------
%\usepackage{url}
\usepackage{hyperref}
\hypersetup{
%draft, % Uncomment to remove all links (useful for printing in black and white)
colorlinks=true, breaklinks=true, % bookmarks=true,
bookmarksnumbered,
%urlcolor=black, linkcolor=RoyalBlue, citecolor=blue, % Color of links
urlcolor=black, linkcolor=black, citecolor=blue, % Color of links
pdftitle={}, % PDF title
pdfauthor={\textcopyright}, % PDF Author
pdfsubject={}, % PDF Subject
pdfkeywords={}, % PDF Keywords
pdfcreator={pdfLaTeX}, % PDF Creator
plainpages=false,
pdfproducer={LaTeX with hyperref and ClassicThesis} % PDF producer
}

