\chapter{Experiments}

In this chapter \ldots 

\section{Quantitative Quality Measures}
\label{sec:measures}

example equation
\begin{align}
    MSE(a,b) &= \frac{1}{N}\sum_{n=1}^N||a_n-b_n||_2^2
    \label{eq:mse}
\end{align}
% % features, model, evaluation


\section{Classification X}

The model described in Sec.~\ref{sec:fancy_procedure} \ldots 

\ldots implementation is based on xy

measure for evaluation Sec.~\ref{sec:measures}, MSE \eqref{eq:mse}


In Fig.~\ref{fig:subfigs} it can be seen \ldots 
Fig.~\ref{fig:subfigs}\subref{subfig:input} the original image, \ldots Fig.~\ref{fig:subfigs}\subref{subfig:detect} \ldots 

\begin{figure}[htb] % "H"ere, "T"op "B"ottom
	\centering
	\subfloat[original]{\includegraphics[width=0.3\textwidth]{dummy_img} \label{subfig:input}}
	\subfloat[detected]{\includegraphics[width=0.25\textwidth]{dummy_img} \label{subfig:detect}}
	\\ % new line if needed
	\subfloat[example 2]{\includegraphics[width=0.48\textwidth]{dummy_img}}
	\subfloat[estimated]{\includegraphics[width=0.48\textwidth]{dummy_img} \label{subfig:ex2}}
	\caption{%
		There should be a short description here. More details in the text: \protect\subref{subfig:input} original image, \protect\subref{subfig:detect} 
		\textcolor{red}{2DO: adapt content, size and description. You can put it all in one row, too.}
	}
	\label{fig:subfigs}
\end{figure}



Results shown in Tab.~\ref{tab:mse_results}. 

\begin{table}[htb]
	\centering
	\caption{Best description ever.} % CAPTION OF TABLES ON TOP
	\label{tab:mse_results}
	\begin{tabular}{cc}
		\toprule 
		M1 & M2 \\ \midrule
		0.5  & \textbf{0}\\
		0.2 &  2\\
		\bottomrule
	\end{tabular}
\end{table}